\documentclass[twocolumn]{IEEEtran}
\IEEEoverridecommandlockouts
% The preceding line is only needed to identify funding in the first footnote. If that is unneeded, please comment it out.
\usepackage{cite}
\usepackage{amsmath,amssymb,amsfonts}
\usepackage{algorithmic}
\usepackage{graphicx}
\usepackage{textcomp}
\usepackage{xcolor}
\def\BibTeX{{\rm B\kern-.05em{\sc i\kern-.025em b}\kern-.08em
    T\kern-.1667em\lower.7ex\hbox{E}\kern-.125emX}}
\begin{document}

\title{Technical Writing and Presentation HW2}

\author{\IEEEauthorblockN{Shyi-Ming Chen, Shao-Hung Yu}
\IEEEauthorblockA{
\textit{*Department of Computer Science and Information Engineering}\\
National Taiwan University of Science and Technology, Taipei, Taiwan\\
12312@gmail.com,123ng@gmail.com}
}

\maketitle

\begin{abstract}
Because the MADM method proposed based on the IVIFV longitude function and the IVIFV weighted average operator in the fuzzy system has some shortcomings, so the district needs to develop a new MADM method to overcome.
\end{abstract}

\begin{IEEEkeywords}
IVIFS,IVIFV,MADM
\end{IEEEkeywords}

\section{Introduction}
In this paper, we propose a novel MADM method based on the proposed novel SF of IVIFVs and the power operator~\cite{An2007}of IVIFVs. The proposed SF of IVIFVs can overcome the drawbacks of the SFs of IVIFVs presented in ~\cite{Bai2013}. Based on the proposed SF of IVIFVs, we propose a new MADM method to overcome the shortcomings of the MADM methods presents in ~\cite{chen1997}. The proposed MADM method is very useful for MADM in IVIF environments


\section{RELATED WORK}
In recent years,some MADM methods~\cite{chen2015}~\cite{chen2017}~\cite{chen2021}~\cite{chen2010}~\cite{Zhao2011}have been proposed based on IVIFSs.However, the MADM methods presented in~\cite{chen2015}~\cite{chen2017}~\cite{chen2021}~\cite{chen2010}~\cite{Zhao2011} have the following drawbacks:

    (1) Chen and Huang~\cite{chen2017_2} pointed out that the MADM method presented in~\cite{chen2015} has the shortcoming of obtaining unreasonable preference orders (POs) of alternatives in some circumstances.
    
    (2) Chen and Han~\cite{chen2018} pointed out that the MADM method presented in~\cite{chen2017} has the drawback that it cannot get the POs of alternatives in some cases since infinite loops occurred.
    
    (3) The MADM method presented in~\cite{chen2021} is not able to distinguish POs of alternatives and obtains unreasonable POs of alternatives in some situations.
    
    (4) Chen and Huang~\cite{chen2017_2} pointed out that the drawback of the MADM method presented in~\cite{chen2010} is that it cannot obtain the POs of alternatives in some circumstances due to the ‘‘division by zero” problem occurred.
    
    (5) Chen and Han~\cite{chen2018} pointed out that the drawback of the MADM method presented in~\cite{Zhao2011} is that it cannot obtain the POs of alternatives in some circumstances due to the ‘‘division by zero” occurred.

Therefore, a new MADM method needs to be developed to overcome the shortcomings of the MADM methods presented in~\cite{chen2015}~\cite{chen2017}~\cite{chen2021}~\cite{chen2010}~\cite{Zhao2011}

\bibliographystyle{IEEEtran}
\bibliography{IEEEabrv,IEEE}


\end{document}
