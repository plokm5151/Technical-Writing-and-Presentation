\documentclass{beamer}
\usepackage{bibstyle/ntust-beamer}
\usepackage{amsmath}
\usepackage{amsfonts}
\usepackage{amssymb}
\usepackage{bibentry}

%%%%%%%%%%%%%%%%%%%%%%%%%%%%%%%%%%%%%%%%%%%%%%%%%%%%%%
%%%%%%%%%%%%%%%%%%%%%%%%%%%%%%%%%%%%%%%%%%%%%%%%%%%%%%

% Info
\title{
    Multiattribute decision making based on novel score function
and the power operator of interval-valued intuitionistic fuzzy
values
}
\author{
    Deng-Cyun Chen
}
\institute{\texttt{
    M11115101@gapps.ntust.edu.tw
}}
\date{
    April 11, 2023 %\today
}

%%%%%%%%%%%%%%%%%%%%%%%%%%%%%%%%%%%%%%%%%%%%%%%%%%%%%%
%%%%%%%%%%%%%%%%%%%%%%%%%%%%%%%%%%%%%%%%%%%%%%%%%%%%%%

\begin{document}
\begin{frame}[plain,noframenumbering]
\placelogofalse
\titlepage
\end{frame}
\placelogotrue

%%%%%%%%%%%%%%%%%%%%%%%%%%%%%%%%%%%%%%%%%%%%%%%%%%%%%%
\begin{frame}[fragile]
	\frametitle{Summary}
	\begin{block}{Research Motivation}{}
    \ \ Because the MADM method proposed based on the IVIFV longitude function and the IVIFV weighted average operator in the fuzzy system has some shortcomings, so the district needs to develop a new MADM method to overcome.
	\end{block}
\end{frame}
%%%%%%%%%%%%%%%%%%%%%%%%%%%%%%%%%%%%%%%%%%%%%%%%%%%%%%
\begin{frame}[fragile]
    \frametitle{Summary}
    \begin{block}{Purpose}
    \ \ The purpose of this paper is to propose a new multiple attribute decision-making (MADM) method based on interval-valued intuitionistic fuzzy values (IVIFVs). The proposed method uses a novel score function and power operator to calculate the weights of alternatives for multi-attribute decision-making. The aim is to overcome the limitations of existing MADM methods and improve the accuracy and efficiency of multi-attribute decision-making.
    \end{block}
\end{frame}
%%%%%%%%%%%%%%%%%%%%%%%%%%%%%%%%%%%%%%%%%%%%%%%%%%%%
\begin{frame}[fragile]
	\frametitle{Introduction}
	\begin{block}{Method}{}
    \ \ Multiple attribute decision-making (MADM) is a method used to evaluate and rank alternatives based on multiple criteria or attributes. MADM methods typically involve assigning weights to each attribute and then calculating a score for each alternative based on its performance on each attribute. The scores are then combined to obtain an overall ranking of the alternatives
	\end{block}
\end{frame}
%%%%%%%%%%%%%%%%%%%%%%%%%%%%%%%%%%%%%%%%%%%%%%%%%%%%
\begin{frame}[fragile]
	\frametitle{Introduction}
	\begin{block}{}
	\ \ In this paper, we propose a novel MADM method based on the proposed novel SF of IVIFVs and the power operator~\cite{An2007}of IVIFVs. The proposed SF of IVIFVs can overcome the drawbacks of the SFs of IVIFVs presented in ~\cite{Bai2013}. Based on the proposed SF of IVIFVs, we propose a new MADM method to overcome the shortcomings of the MADM methods presents in ~\cite{chen1997}. The proposed MADM method is very useful for MADM in IVIF environments
	\end{block}
\end{frame}
%%%%%%%%%%%%%%%%%%%%%%%%%%%%%%%%%%%%%%%%%%%%%%%%%
\begin{frame}[fragile]
	\frametitle{Related work}
	\begin{block}{}
	\ \ In recent years,some MADM methods~\cite{chen2015}~\cite{chen2017}~\cite{chen2021}~\cite{chen2010}~\cite{Zhao2011}have been proposed based on IVIFSs.However, the MADM methods presented in~\cite{chen2015}~\cite{chen2017}~\cite{chen2021}~\cite{chen2010}~\cite{Zhao2011} have the following drawbacks:

 
    (1) Chen and Huang~\cite{chen2017_2} pointed out that the MADM method presented in~\cite{chen2015} has the shortcoming of obtaining unreasonable preference orders (POs) of alternatives in some circumstances.
    
    (2) Chen and Han~\cite{chen2018} pointed out that the MADM method presented in~\cite{chen2017} has the drawback that it cannot get the POs of alternatives in some cases since infinite loops occurred.
	\end{block}
\end{frame}
%%%%%%%%%%%%%%%%%%%%%%%%%%%%%%%%%%%%%%%%%%%%%%%%%
\begin{frame}[fragile]
	\frametitle{Related work}
	\begin{block}{}
    
    (3) The MADM method presented in~\cite{chen2021} is not able to distinguish POs of alternatives and obtains unreasonable POs of alternatives in some situations.
    
    (4) Chen and Huang~\cite{chen2017_2} pointed out that the drawback of the MADM method presented in~\cite{chen2010} is that it cannot obtain the POs of alternatives in some circumstances due to the ‘‘division by zero” problem occurred.
    
    (5) Chen and Han~\cite{chen2018} pointed out that the drawback of the MADM method presented in~\cite{Zhao2011} is that it cannot obtain the POs of alternatives in some circumstances due to the ‘‘division by zero” occurred.

Therefore, a new MADM method needs to be developed to overcome the shortcomings of the MADM methods presented in~\cite{chen2015}~\cite{chen2017}~\cite{chen2021}~\cite{chen2010}~\cite{Zhao2011}
	\end{block}
\end{frame}
%%%%%%%%%%%%%%%%%%%%%%%%%%%%%%%%%%%%%%%%%%%%%%%%%
\begin{frame}[fragile]
	\frametitle{Method}
        \begin{block}{Step1}
	\ \ Transform each IVIF weight of a criterion into a crisp weight using a proposed formula. This step allows for a better reflection of the decision maker's relative importance of various criteria.
        \end{block}
    
        
\end{frame}
%%%%%%%%%%%%%%%%%%%%%%%%%%%%%%%%%%%%%%%%%%%%%%%%%

\begin{frame}[fragile]
	\frametitle{Method}
        \begin{block}{Step2}
        \ \ Evaluate each criterion to determine its relative importance in the decision-making process. This involves assigning IVIF weights to each criterion.
        \end{block}
        
\end{frame}
%%%%%%%%%%%%%%%%%%%%%%%%%%%%%%%%%%%%%%%%%%%%%%%%%

\begin{frame}[fragile]
	\frametitle{Method}
        \begin{block}{Step3}
        \ \ Calculate the score of each alternative on each criterion using a new score function. This step involves using the proposed score function to calculate the score of each alternative on various criteria.
        \end{block}
        
\end{frame}
%%%%%%%%%%%%%%%%%%%%%%%%%%%%%%%%%%%%%%%%%%%%%%%%%
\begin{frame}[fragile]
	\frametitle{Method}
        \begin{block}{Step4}
        \ \ Convert scores into weights using a power operator. This step involves using a power operator to convert scores into weights, which better reflect the relative importance of various criteria.
        \end{block}
        
\end{frame}
%%%%%%%%%%%%%%%%%%%%%%%%%%%%%%%%%%%%%%%%%%%%%%%%%

\begin{frame}[fragile]
	\frametitle{Method}
        \begin{block}{Step5}
        \ \ Multiply the weights of each criterion and alternative to obtain final scores. This step involves multiplying the weights of each criterion and alternative to obtain final scores for each alternative. Based on these scores, alternatives can be ranked to obtain the optimal solution.
        \end{block}
        
\end{frame}
%%%%%%%%%%%%%%%%%%%%%%%%%%%%%%%%%%%%%%%%%%%%%%%%%
\begin{frame}[fragile]
	\frametitle{Conclusions}
        \ \ In this paper, a new multi-criteria decision-making method based on IVIF weights and a proposed score function was presented. The proposed method allows for a better reflection of the decision maker's relative importance of various criteria, which improves the efficiency and accuracy of the decision-making process. The results of the case study demonstrate that the proposed method outperforms other existing methods in terms of accuracy and efficiency. Furthermore, the proposed method is flexible and can be applied to various decision-making problems with different criteria. In conclusion, the proposed method provides an effective approach for solving complex decision-making problems with multiple criteria, and has great potential for.
        
\end{frame}
%%%%%%%%%%%%%%%%%%%%%%%%%%%%%%%%%%%%%%%%%%%%%%%%%%%%%%

\setbeamertemplate{bibliography item}[text]
\frame[allowframebreaks]{
	\frametitle{Reference}
	\fontsize{9pt}{13}\selectfont
    \bibliographystyle{bibstyle/IEEEtran}
    \bibliography{bibstyle/IEEEabrv,refs}
}

\end{document}
